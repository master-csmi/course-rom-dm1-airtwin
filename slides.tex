\documentclass{beamer}
\usetheme{Copenhagen}
\usepackage{graphicx}
% \usepackage[ruled,vlined,english]{algorithm2e}
% \renewcommand{\thealgocf}{}
% \usepackage{hyperref}


\author[Jumeau numérique: Environnement]{\large Hanna CHETOUANE, Narmimane Zaouache \\ \vspace{-0.8cm} \date{\today}}
\title[Microclimat urbain]{\textbf{Jumeau numérique dans l'environnement:} \\ Microclimat urbain}



\begin{document}

\begin{frame}
    \titlepage
    \vspace{-0.4cm}
    \begin{center}
        UFR of Mathematics and Informatics - University of Strasbourg 
        \\[0.2cm] 
        \textit{Problématique: A quoi sert un jumeau numérique/ comment réduire les îlots de chaleur urbains, prévisions, surveillance, améliorer \\
        Comment la végétation, la géométrie et les matériaux urbains influencent-ils sur le micro-climat urbain ?} % quoi choisir comme problématique ?
    \end{center}
\end{frame} 


\begin{frame}{Plan}
    \tableofcontents    
\end{frame}


\begin{frame}{Qu'est-ce qu'un jumeau numérique ?}
    \rightarrow réplique virtuelle et dynamoque d'un système physique % ou autre ?
    \rightarrow Données qui servent à modéliser des phénomènes, simuler des scénarios et ainsi prédire et optimiser un système réel 
    \textbf{Objectifs en lien avec notre problématique:}
    \begin{itemize}
        \item Suivre l'évolution du microclimat urbain % Surveillance
        \item Ajuster le modèle avec des données réelles en continue % callibration
        \item Tester des scénarios d'aménagement urbain (stratégie de végétalisation, matériaux, géométrie) % optimisation
        \item Anticiper les changements climatiques et leurs impacts (pollution, chaleur) % prédiction
    \end{itemize}
    Défi actuel: projet JNFT en France avec IGN, Cerema, Inria
\end{frame}


\begin{frame}{Problématique et contexte} % ou juste contexte ?
    %besoin métieur/scientifique
    Comprendre les interactions entre les facteurs urbains et le microclimat
    \rightarrow
    optimiser les aménagements urbains pour réduire les îlots de chaleur urbains
    aide aux décisions de nouvelles infrastructures, végétation, matériaux
    % périmètre fonctionnel + KPI ciblés
    Modélisation d'une zone urbaine localisée
    Données météorologiques (température, humidité, vent), données urbaines (batiments, végétation, matériaux) % diagramme
\end{frame}


\begin{frame}{Architecture et pipeline}
    % shéma
\end{frame}


\begin{frame}{Méthodes}
    
\end{frame}


\begin{frame}{Données et instrumentation}
    % tableau
\end{frame}


\begin{frame}{Vérification et validation / UQ}
    
\end{frame}


\begin{frame}{Transfert et déploiement}
    
\end{frame}


\begin{frame}{Perspectives et limites}
    
\end{frame}


\begin{frame}{Bibliographie}
    https://www.ign.fr/institut/un-jumeau-numerique-de-la-france-pour-piloter-la-transition-ecologique
    https://cnig.gouv.fr/IMG/pdf/2025.01_20_jnft_presentation_cnig_pole_territoires.pdf
    % Fondateurs de la démarche: IGN, Cerema, Inria
    % slide 10 -> 30% pour les aménagements durables
\end{frame}



\end{document}